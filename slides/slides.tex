\documentclass[9pt,xcolor=dvipsnames]{beamer}
\usepackage[all]{xy}
\usepackage{animate}

\usepackage{amsmath}
\usepackage{amsfonts}
\usepackage{amsthm}

\newtheorem{proposition}{Proposition}
\newtheorem{remark}{Remark}
%\newtheorem{theorem}{Theorem}

% for figs from GeoGebra
\usepackage[utf8]{inputenc}
\usepackage{pgf,tikz}
\usetikzlibrary{arrows}
% end GeoGebra
\usepackage{listings}
\definecolor{codegreen}{rgb}{0,0.6,0}
\definecolor{codegray}{rgb}{0.5,0.5,0.5}
\definecolor{codepurple}{rgb}{0.58,0,0.82}
\definecolor{backcolour}{rgb}{0.95,0.95,0.92}
 
\lstdefinestyle{pystyle}{
    backgroundcolor=\color{backcolour},   
    commentstyle=\color{codegreen},
    keywordstyle=\color{magenta},
    numberstyle=\tiny\color{codegray},
    stringstyle=\color{codepurple},
    basicstyle=\footnotesize,
    breakatwhitespace=false,         
    breaklines=true,                 
    captionpos=b,                    
    keepspaces=true,                 
    numbers=left,                    
    numbersep=5pt,                  
    showspaces=false,                
    showstringspaces=false,
    showtabs=false,                  
    tabsize=2
}
\lstset{style=pystyle}

\usepackage{wrapfig}

%\usepackage[pdftex]{graphicx}
%\usepackage{epstopdf}
%\usepackage{graphicx}
\newtheorem{algorithm}[theorem]{Algorithm}
\newcommand{\Ropt}{\mathcal{R}}
\newcommand{\semfootnote}[1]{\let\thefootnote\relax\footnotetext{#1}}
%%%%%%%%%%%%%%%%%%%%%%%%%%%%%%%%%%%%%%%%%%%%%%%%%%%%%%%%%%
\definecolor{MyBlue}{rgb}{0.1,0.3,1} 
\definecolor{MyBlue2}{rgb}{0.1,0.3,0.6}
\definecolor{orange}{rgb}{0.5,0.5,0.}
\definecolor{darkblue}{rgb}{.1,.1,.1}
\definecolor{darkgreen}{rgb}{0.,.4,0.}

\mode<presentation>
{
\usetheme{Pittsburgh}
\usecolortheme{seahorse}
\setbeamertemplate{blocks}[rounded]
\setbeamercolor{title}{bg=MyBlue2,fg=white}
\setbeamercolor{frametitle}{bg=MyBlue2,fg=white}
%\setbeamercolor{section number projected}{bg=gray!50,fg=black}
\setbeamercolor{section in toc}{fg=black}
\setbeamercolor{block title}{fg=black,bg=darkgreen!70}
\setbeamercolor{block body}{fg=black,bg=darkgreen!10}
%\setbeamercolor{block title alerted}{fg=red,bg=darkgreen!40}
%\setbeamerfont{block title}{series=\bfseries, bg=Myblue}
\setbeamerfont{title}{size=\LARGE}
%
%\usefonttheme{serif}
}
%%%%%%%%%%%%%%%%%%%%%%%%%%%%%%%%%%%%%%%%%%%%%%%%%%%%%%%%%%
%\usepackage{remreset}
%\makeatletter
%\@removefromreset{subsection}{section}
%\makeatother
%\setcounter{subsection}{-1}
%%%%%%%%%%%%%%%%%%%%%%%%%%%%%%%%%%%%%%%%%%%%%%%%%%%%%%%%%%
%% extra definitions
%% September 2, 2013

% math definitions
\newcommand{\OM}{\Omega}
\newcommand{\RE}{\mathbb{R}}
\newcommand{\PO}{\mathbb{P}}
\newcommand{\NA}{\mathbb{N}}
\newcommand{\ra}{\mathfrak{r}}
\newcommand{\Dt}{\Delta t}
\newcommand{\Dx}{\Delta x}
\newcommand{\Dif}[2]{\frac{d {#1}}{d {#2}}}
\newcommand{\PDif}[2]{\frac{\partial {#1}}{\partial {#2}}}
%
\newcommand{\Lsp}{\textrm{L}}
\newcommand{\Hsp}{\textrm{H}}
\newcommand{\Tsp}{\textrm{T}}
%
\newcommand{\Uh}{\mathbf{u}_h}
\newcommand{\Uhbar}{\mathbf{\bar{u}}_h}
\newcommand{\Ufbar}{\mathbf{\bar{u}}_f}
\newcommand{\Ucbar}{\mathbf{\bar{u}}_c}
\newcommand{\Uf}{\mathbf{u}_f}
\newcommand{\Uc}{\mathbf{u}_c}
\newcommand{\Eh}{\mathbf{e}_h}
\newcommand{\Ef}{\mathbf{e}_f}
\newcommand{\Ec}{\mathbf{e}_c}
\newcommand{\DI}{\mathcal{D}}
\newcommand{\Arr}[2]{
	\left(
		\begin{array}{c}
		{#1} 
		%
		\\
		%
		{#2}
		\end{array}
	\right)
}
\newcommand{\Arrtri}[3]{
	\left(
		\begin{array}{c}
		{#1} 
		%
		\\
		%
		{#2}
		\\
		%
		{#3}		
		\end{array}
	\right)
}
\newcommand{\ARR}[4]{
	\left(
		\begin{array}{c}
		{#1} 
		%
		\\
		%
		{#2}
		\\
		{#3} 
		%
		\\
		%
		{#4}		
		\end{array}
	\right)
}
\newcommand{\MAT}[4]{
	\left[
		\begin{array}{c|c}
		{#1} 
		%
		&
		%
		{#2}
		\\ \hline
		{#3} 
		%
		&
		%
		{#4}		
		\end{array}
	\right]
}
\newcommand{\MATT}[4]{
	\left[
		\begin{array}{cc}
		{#1} 
		%
		&
		%
		{#2}
		\\
		{#3} 
		%
		&
		%
		{#4}		
		\end{array}
	\right]
}
\newcommand{\MATnine}[9]{
	\left[
		\begin{array}{ccc}
		{#1} & {#2} & {#3}
		\\
		{#4} & {#5} & {#6}
		%
		\\
		%
		{#7} & {#8} & {#9}		
		\end{array}
	\right]
}

\newcommand{\tA}{\tilde{A}}
\newcommand{\tB}{\tilde{B}}
\newcommand{\Ont}{\mathcal{O}}
\newcommand{\half}{\frac{1}{2}}

% Finite element, DG
%% jump - average operators
\newcommand{\average}[1]{\ensuremath{\lbrace\!\!\lbrace#1\rbrace\!\!\rbrace} } 
\newcommand{\jump}[1]{\ensuremath{[\![#1]\!]} }

\newcommand{\jumpF}[1]{\ensuremath{[\![\![#1]\!]\!]} }
\newcommand{\averageF}[1]{\ensuremath{\lbrace\!\lbrace\!\lbrace#1\rbrace\!\rbrace\!\rbrace} } 


\newcommand{\diff}{\varepsilon} %usata
%
\newcommand{\Th}{\mathcal{T}_h} %usata
\newcommand{\Thi}[1]{\mathcal{T}_{h,#1}} %usata
\newcommand{\Ti}[1]{\mathcal{T}_{#1}} %usata
%
\newcommand{\TAU}{{\boldsymbol \tau}}
\newcommand{\SIG}{{\boldsymbol \sigma} }
\newcommand{\BT}{{\boldsymbol \beta}}
\newcommand{\NOR}{{\boldsymbol n} }
\newcommand{\fB}{{\boldsymbol f} }
\newcommand{\vB}{{\boldsymbol v} }
\newcommand{\uB}{{\boldsymbol u} }
\newcommand{\wB}{{\boldsymbol w} }
\newcommand{\BO}[1]{{\boldsymbol #1} }
\newcommand{\lB}{{\boldsymbol \lambda} }
\newcommand{\phiB}{{\boldsymbol \varphi} }
\newcommand{\UDOF}{\underline{\boldsymbol u} }
\newcommand{\SDOF}{\underline{\boldsymbol \sigma} }
\newcommand{\Bopt}{\mathcal{B}}
\newcommand{\Hopt}{\mathcal{H}}
\newcommand{\Zopt}{{\boldsymbol \theta}}
%
\newcommand{\EPS}{ \mathcal{E} }
%
\newcommand{\GAMF}{ \Gamma^{ \mathtt{f} } }
\newcommand{\GAMI}{ \Gamma^{ 0 } }
\newcommand{\GAM}{ \Gamma }

%
\newcommand{\dotV}[2]{\left( {#1} , {#2} \right) }
\newcommand{\dotS}[2]{\left\langle #1 , #2 \right\rangle }

\newcommand{\Nel}{N_{\text{el} }}
\newcommand{\hmax}{h_{\text{max} }}
%\newcommand{\trin}[1]{{\left\vert\kern-0.25ex\left\vert\kern-0.25ex\left\vert #1 
%   \right\vert\kern-0.25ex\right\vert\kern-0.25ex\right\vert}}
\newcommand{\DGnorm}[2]{\Vert {#1} \Vert_{\textrm{\tt DG}{#2}} }
\newcommand{\Bnorm}[2]{\Vert {#1} \Vert_{{\tt HDG} #2} }
\newcommand{\Lnorm}[1]{{ \Vert #1 \Vert }}
\newcommand{\semin}[1]{\left| #1 \right|}

%\newcommand*{\defeq}{\mathrel{\vcenter{\baselineskip0.5ex \lineskiplimit0pt
%                                  \hbox{\scriptsize.}\hbox{\scriptsize.}}}=}
\newcommand{\norm}[1]{{ \Vert #1 \Vert }}


%
% Tikz configuration
\tikzset{
  every overlay node/.style={
    %draw=black,fill=white,rounded corners,
    anchor=north west, inner sep=0pt,
  },
}
% Usage:
% \tikzoverlay at (-1cm,-5cm) {content};
% or
% \tikzoverlay[text width=5cm] at (-1cm,-5cm) {content};
\def\tikzoverlay{%
   \tikz[remember picture, overlay]\node[every overlay node]
}%
%
\begin{document}
%
\date{Nov. 25, 2022}

\author[S. Hajian]{
	{Soheil Hajian}
	}
%
\institute[]
	{
	  %\inst{1}
	\vspace{1cm}
	{\normalsize\texttt{soheil.hajian@outlook.com}}
	}

\setbeamertemplate{navigation symbols}{} % get the rid of navigation symbols
\setbeamertemplate{footline}[frame number]
%
\title[]{Software testing: What, Why and How}

%
%\subtitle[]{Discontinuous Galerkin method}
%
%%%%%%%%%%%%%%%%%%%%%%%%%%%%%%%%%%%%%%%%%%%%%%%%%%%%%%%%%%
%
\begin{frame}[plain]
   %% \tikz [remember picture,overlay]
   %%  \node at
   %%      ([yshift=4.cm, xshift=-3cm]current page.south) 
   %%      %or: (current page.center)
   %%      {\includegraphics[]{figs/mesh/ddmesh5.jpg}};

\begin{titlepage}
\end{titlepage}
\vfill
\end{frame}

\begin{frame}{Outline}
    \tableofcontents
\end{frame}
%

\section{What is software testing, and why is it important?}
\begin{frame}
  \frametitle{What is software testing, and why is it important?}
  \begin{itemize}
  \item What is testing?
    \begin{itemize}
    \item \textbf{Software testing} is the process of verfiying that a
      software does what it is supposed to do.
    \item Verifiability implies some knowledge of \textbf{requirements} for that software
      or its components.
    \end{itemize}
    \pause 
  \item Why is testing important?
    \begin{itemize}
      \item A test is a certificate that the software (or a piece of
        it) behaves as expected.
      \item Ensures that future updates to the software do not break
        old behavior.
      \item Gives a sense of code reliability. 
    \end{itemize}
  \end{itemize}

\end{frame}

\section{Different levels of testing}
\begin{frame}
  \frametitle{Different levels of testing}
  It is common practice to split the testing into different
  levels. The most common splitting contains: \textbf{unit-}, \textbf{integration-}, and
  \textbf{system testing}.
  \vspace{1cm}
  \begin{center}
    \includegraphics[scale=0.3]{figs/levels.png}
  \end{center}
\end{frame}

\begin{frame}
  \frametitle{Unit testing}
  \begin{itemize}
    \item \textbf{Unit testing} is usually referred to testing an
      \textbf{individual component} of the code, e.g., functions or methods of a class.
    \item \textbf{Example:} Write test for a function that takes a float as
      input and returns the square of the input.
      \pause
      \begin{itemize} 
      \item Code:
        \lstinputlisting[language=Python]{../testing_exercises/some_functions/square.py}
        \pause
      \item Test:
        \lstinputlisting[language=Python]{../tests/unit_tests/some_functions/test_square.py}
      \end{itemize}
  \end{itemize}
\end{frame}


\begin{frame}
  \frametitle{Integration testing}
  \begin{overlayarea}{\textwidth}{\textheight}
  \begin{itemize}
    \item \textbf{Integration testing} verifies if a group of
      smaller units within a software works as expected.
    \item \textbf{Example:} Write test for a function that takes
      credentials to connect to a database, runs a query and return the
      result as a dataframe.
      \begin{itemize}
      \item<only@1> Flowchart:
        \vspace{1cm}
        \begin{center}
          \includegraphics[scale=0.4]{figs/get_table.png}
        \end{center}
      \item<only@2> Code:
          \lstinputlisting[language=Python]{../testing_exercises/some_reader/read_db.py}
        \item<only@3> Test:
          \lstinputlisting[language=Python]{../tests/integration_tests/some_reader/test_reader_db.py}
      \end{itemize}
  \end{itemize}
  \end{overlayarea}
\end{frame}

\begin{frame}
  \frametitle{System testing}
  \begin{overlayarea}{\textwidth}{\textheight}
  \begin{itemize}
    \item \textbf{System testing} checks if the software works as
      expected \textbf{technically} and \textbf{end-to-end}.
    \item \textbf{Example:} Write a test for a software that takes an
      input (credentials etc.) as a file, fetches data from a
      database, perform some logic and dump the result as a json file.
\begin{table}[ht]
  \begin{minipage}[t]{0.45\linewidth} \centering
    \raisebox{-\height}{
      \includegraphics[scale=0.3]{figs/end_to_end.png}
    }    
  \end{minipage}	
  %
  %\hspace{-0.5cm}
  %\hfill
  %
  \begin{minipage}[t]{0.35\linewidth}
    \pause
    Note: Here we check that everything works \textbf{technically} and not the
    core logic.
  \end{minipage}	
\end{table}
  \end{itemize}
  \end{overlayarea}
\end{frame}

\section{Test Driven Development}

\begin{frame}
  \frametitle{Test Driven Development}
  \begin{overlayarea}{\textwidth}{\textheight}
  \begin{itemize}
    \item<1-2>   ``Test-driven development (TDD) is a \textbf{software development process}
  relying on software requirements being \textbf{converted} to \textbf{test cases}
  before software is fully developed ...'' - Wikipedia.
    \item Steps:
    \begin{enumerate}
    \item<1-3> Add tests: focus on requirements before writing the actual
      code.
    \item<1-2,4> Run all the tests: they should fail.
    \item<1-2,5> Write the simplest code that pass the test.
    \item<1-2,6> Run all the tests: All tests should pass now.
    \item<1-2,7> Refactor the code as needed.
    \end{enumerate}
  \item<only@2> TDD is often used for unit-testing.
  \item<only@2> It uses \textbf{\color{red}mocks} to represent the
    outside world.
  \item<3->
    \fbox{\begin{minipage}{0.9\textwidth}
        \textbf{Example:} Write test for a function that takes a float as
               input and returns the square of the input.
    \end{minipage}}
  \item<only@3> Test:
      \lstinputlisting[language=Python]{../tests/unit_tests/some_functions/test_square_tdd.py}
    \item<only@4> Test fails for this code:
      \lstinputlisting[language=Python]{../testing_exercises/some_functions/square.py}      
    \item<5-6> Code:
      \lstinputlisting[language=Python]{../testing_exercises/some_functions/square_tdd.py}      

  \end{itemize}
  \end{overlayarea}
\end{frame}

\section{Tips and tricks}
\begin{frame}
  \frametitle{Tips and tricks for Python}
  \begin{overlayarea}{\textwidth}{\textheight}
  \begin{itemize}
    \item Use {\color{red}\tt mock} library in {\tt python} to simulate
      outside world, e.g., databases and end-points, during
      {\color{red}unit-testing}.
    \item<only@2>[] Example: Write test for a function that takes
      credentials to connect to a database, runs a query and return the
      result as a dataframe.
      \lstinputlisting[language=Python]{
        ../tests/unit_tests/some_reader/test_reader_db.py}
      \item<3-> There are two packages for testing: {\tt pytest} and
        {\tt unittest}.

        (personal opinion: {\tt unittest} is more
        explicit and verbose to read)
          \item<4-> If your code writes files into the disk,
            create temporary paths for it during the test; DO NOT
            WRITE INTO A HARD-CODED PATH.
          \item<only@4>[] \lstinputlisting[language=Python]{../tests/unit_tests/some_writer/test_writer.py}
        \item<5-> Tests should be deterministic: if your code relies
          on random generators, fix the {\tt \color{red}seed} number during the test.
        \item<6-> In order to simulate a real database for integration
          test, try to use {\color{red}a test container} for it. (not
          always possible)
          \item<7-> Using {\tt Given, When, Then} pattern improves
            readability of the test. (optional)
  \end{itemize}
  \end{overlayarea}
\end{frame}

\section{Exercises}
\begin{frame}
  \frametitle{Exercises - Unit testing}
  \begin{itemize}
  \item Write test for a function that
    \begin{itemize}
    \item takes two inputs,
    \item each input can be either a string or {\tt None},
    \item if both strings are present, concatenate them and return
      the value,
    \item if either of them is {\tt None}, return {\tt None}. 
    \end{itemize}
  \item Implement the function under
    \url{testing_exercises/some_functions/concatenate.py}
  \item Use TDD framework.
  \end{itemize}
\end{frame}

\begin{frame}
  \frametitle{Exercises - Unit testing II}
  \begin{itemize}
  \item Write test for a function that
    \begin{itemize}
    \item takes arbitrary number of inputs,
    \item accepts only integers,
    \item returns their sum.
    \end{itemize}
  \item Use {\tt *args} to pass arbitrary number of inputs.
    \item Use TDD framework.
  \end{itemize}
\end{frame}

\end{document}
%
